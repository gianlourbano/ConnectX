\chapter{Conclusioni}

Come possiamo vedere dai grafici, entrambi \verb!GranVisirSchillaci! e \verb!Schillaci_L1! hanno performance molto simili, visitando un numero di nodi affine.
La differenza più grande sta nel numero di nodi riutilizzati, in quanto \verb!GranVisirSchillaci! usa una transposition table più efficiente e che gestisce meglio le collisioni.
Grazie a quest'ultima infatti visitando lo stesso numero di nodi è in grado di vincere di più, come dimostrato dalla tabella \ref{table:2}.

\section {Altre possibili migliorie}

Come detto in precedenza, ci sono molte altre possibili migliore che si possono apportare all'AI per affinarne le performance.
Tra queste:
\begin{itemize}
    \item la ricerca quiescente, che permette di evitare il problema dell'effetto orizzonte;
    \item varie euristiche come le killer moves e la history heuristic;
    \item implementazioni più efficienti della transposition table, non utilizzando le \verb!HashMap! di Java;
    \item implementazioni più efficienti della funzione di valutazione euristica;
    \item machine learning, addestrando l'AI contro se stessa;
    \item opening book, che permette di memorizzare le mosse di apertura più vantaggiose;
    \item Monte Carlo Tree Search, un metodo di ricerca basato su simulazioni;
\end{itemize}

